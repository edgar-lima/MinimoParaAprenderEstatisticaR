% Options for packages loaded elsewhere
\PassOptionsToPackage{unicode}{hyperref}
\PassOptionsToPackage{hyphens}{url}
%
\documentclass[
]{article}
\title{O mínimo que você precisa saber para aprender estatística no R.}
\author{Prof.~MSc. Edgar Luiz de Lima}
\date{19/03/2022}

\usepackage{amsmath,amssymb}
\usepackage{lmodern}
\usepackage{iftex}
\ifPDFTeX
  \usepackage[T1]{fontenc}
  \usepackage[utf8]{inputenc}
  \usepackage{textcomp} % provide euro and other symbols
\else % if luatex or xetex
  \usepackage{unicode-math}
  \defaultfontfeatures{Scale=MatchLowercase}
  \defaultfontfeatures[\rmfamily]{Ligatures=TeX,Scale=1}
\fi
% Use upquote if available, for straight quotes in verbatim environments
\IfFileExists{upquote.sty}{\usepackage{upquote}}{}
\IfFileExists{microtype.sty}{% use microtype if available
  \usepackage[]{microtype}
  \UseMicrotypeSet[protrusion]{basicmath} % disable protrusion for tt fonts
}{}
\makeatletter
\@ifundefined{KOMAClassName}{% if non-KOMA class
  \IfFileExists{parskip.sty}{%
    \usepackage{parskip}
  }{% else
    \setlength{\parindent}{0pt}
    \setlength{\parskip}{6pt plus 2pt minus 1pt}}
}{% if KOMA class
  \KOMAoptions{parskip=half}}
\makeatother
\usepackage{xcolor}
\IfFileExists{xurl.sty}{\usepackage{xurl}}{} % add URL line breaks if available
\IfFileExists{bookmark.sty}{\usepackage{bookmark}}{\usepackage{hyperref}}
\hypersetup{
  pdftitle={O mínimo que você precisa saber para aprender estatística no R.},
  pdfauthor={Prof.~MSc. Edgar Luiz de Lima},
  hidelinks,
  pdfcreator={LaTeX via pandoc}}
\urlstyle{same} % disable monospaced font for URLs
\usepackage[margin=1in]{geometry}
\usepackage{color}
\usepackage{fancyvrb}
\newcommand{\VerbBar}{|}
\newcommand{\VERB}{\Verb[commandchars=\\\{\}]}
\DefineVerbatimEnvironment{Highlighting}{Verbatim}{commandchars=\\\{\}}
% Add ',fontsize=\small' for more characters per line
\usepackage{framed}
\definecolor{shadecolor}{RGB}{248,248,248}
\newenvironment{Shaded}{\begin{snugshade}}{\end{snugshade}}
\newcommand{\AlertTok}[1]{\textcolor[rgb]{0.94,0.16,0.16}{#1}}
\newcommand{\AnnotationTok}[1]{\textcolor[rgb]{0.56,0.35,0.01}{\textbf{\textit{#1}}}}
\newcommand{\AttributeTok}[1]{\textcolor[rgb]{0.77,0.63,0.00}{#1}}
\newcommand{\BaseNTok}[1]{\textcolor[rgb]{0.00,0.00,0.81}{#1}}
\newcommand{\BuiltInTok}[1]{#1}
\newcommand{\CharTok}[1]{\textcolor[rgb]{0.31,0.60,0.02}{#1}}
\newcommand{\CommentTok}[1]{\textcolor[rgb]{0.56,0.35,0.01}{\textit{#1}}}
\newcommand{\CommentVarTok}[1]{\textcolor[rgb]{0.56,0.35,0.01}{\textbf{\textit{#1}}}}
\newcommand{\ConstantTok}[1]{\textcolor[rgb]{0.00,0.00,0.00}{#1}}
\newcommand{\ControlFlowTok}[1]{\textcolor[rgb]{0.13,0.29,0.53}{\textbf{#1}}}
\newcommand{\DataTypeTok}[1]{\textcolor[rgb]{0.13,0.29,0.53}{#1}}
\newcommand{\DecValTok}[1]{\textcolor[rgb]{0.00,0.00,0.81}{#1}}
\newcommand{\DocumentationTok}[1]{\textcolor[rgb]{0.56,0.35,0.01}{\textbf{\textit{#1}}}}
\newcommand{\ErrorTok}[1]{\textcolor[rgb]{0.64,0.00,0.00}{\textbf{#1}}}
\newcommand{\ExtensionTok}[1]{#1}
\newcommand{\FloatTok}[1]{\textcolor[rgb]{0.00,0.00,0.81}{#1}}
\newcommand{\FunctionTok}[1]{\textcolor[rgb]{0.00,0.00,0.00}{#1}}
\newcommand{\ImportTok}[1]{#1}
\newcommand{\InformationTok}[1]{\textcolor[rgb]{0.56,0.35,0.01}{\textbf{\textit{#1}}}}
\newcommand{\KeywordTok}[1]{\textcolor[rgb]{0.13,0.29,0.53}{\textbf{#1}}}
\newcommand{\NormalTok}[1]{#1}
\newcommand{\OperatorTok}[1]{\textcolor[rgb]{0.81,0.36,0.00}{\textbf{#1}}}
\newcommand{\OtherTok}[1]{\textcolor[rgb]{0.56,0.35,0.01}{#1}}
\newcommand{\PreprocessorTok}[1]{\textcolor[rgb]{0.56,0.35,0.01}{\textit{#1}}}
\newcommand{\RegionMarkerTok}[1]{#1}
\newcommand{\SpecialCharTok}[1]{\textcolor[rgb]{0.00,0.00,0.00}{#1}}
\newcommand{\SpecialStringTok}[1]{\textcolor[rgb]{0.31,0.60,0.02}{#1}}
\newcommand{\StringTok}[1]{\textcolor[rgb]{0.31,0.60,0.02}{#1}}
\newcommand{\VariableTok}[1]{\textcolor[rgb]{0.00,0.00,0.00}{#1}}
\newcommand{\VerbatimStringTok}[1]{\textcolor[rgb]{0.31,0.60,0.02}{#1}}
\newcommand{\WarningTok}[1]{\textcolor[rgb]{0.56,0.35,0.01}{\textbf{\textit{#1}}}}
\usepackage{graphicx}
\makeatletter
\def\maxwidth{\ifdim\Gin@nat@width>\linewidth\linewidth\else\Gin@nat@width\fi}
\def\maxheight{\ifdim\Gin@nat@height>\textheight\textheight\else\Gin@nat@height\fi}
\makeatother
% Scale images if necessary, so that they will not overflow the page
% margins by default, and it is still possible to overwrite the defaults
% using explicit options in \includegraphics[width, height, ...]{}
\setkeys{Gin}{width=\maxwidth,height=\maxheight,keepaspectratio}
% Set default figure placement to htbp
\makeatletter
\def\fps@figure{htbp}
\makeatother
\setlength{\emergencystretch}{3em} % prevent overfull lines
\providecommand{\tightlist}{%
  \setlength{\itemsep}{0pt}\setlength{\parskip}{0pt}}
\setcounter{secnumdepth}{-\maxdimen} % remove section numbering
\ifLuaTeX
  \usepackage{selnolig}  % disable illegal ligatures
\fi

\begin{document}
\maketitle

\hypertarget{muxf3dulo-1---iniciando-os-trabalhos}{%
\section{Módulo 1 - Iniciando os
trabalhos}\label{muxf3dulo-1---iniciando-os-trabalhos}}

\hypertarget{muxf3dulo-2---variuxe1veis-e-tipo-de-dados}{%
\section{Módulo 2 - Variáveis e tipo de
dados}\label{muxf3dulo-2---variuxe1veis-e-tipo-de-dados}}

\hypertarget{usando-o-r-como-calculadora}{%
\subsection{Usando o R como
calculadora}\label{usando-o-r-como-calculadora}}

Podemos utilizar o R para realizar operações básicas de som +, subtração
-, multiplicação *, divisão / e exponiação \^{}.

\begin{Shaded}
\begin{Highlighting}[]
\DecValTok{2}\SpecialCharTok{+}\DecValTok{2}
\end{Highlighting}
\end{Shaded}

\begin{verbatim}
## [1] 4
\end{verbatim}

\begin{Shaded}
\begin{Highlighting}[]
\DecValTok{2}\SpecialCharTok{*}\DecValTok{2}
\end{Highlighting}
\end{Shaded}

\begin{verbatim}
## [1] 4
\end{verbatim}

\begin{Shaded}
\begin{Highlighting}[]
\DecValTok{2}\SpecialCharTok{/}\DecValTok{2}
\end{Highlighting}
\end{Shaded}

\begin{verbatim}
## [1] 1
\end{verbatim}

\begin{Shaded}
\begin{Highlighting}[]
\DecValTok{2{-}2}
\end{Highlighting}
\end{Shaded}

\begin{verbatim}
## [1] 0
\end{verbatim}

\begin{Shaded}
\begin{Highlighting}[]
\DecValTok{3}\SpecialCharTok{\^{}}\DecValTok{2}
\end{Highlighting}
\end{Shaded}

\begin{verbatim}
## [1] 9
\end{verbatim}

Também podemos salvar os resultados dentro de um objeto, por exemplo:
vamos elevar 3 ao quadrado e salvar dentro de objeto chamado A.

\begin{Shaded}
\begin{Highlighting}[]
\NormalTok{A}\OtherTok{\textless{}{-}} \DecValTok{3}\SpecialCharTok{\^{}}\DecValTok{2}
\end{Highlighting}
\end{Shaded}

Note que o R não mostra o resultado, então temos que pedir para o R nos
mostrar o resultado da operação.

Outra coisa importante é que o R diferencia A de a, o R interpreta
letras maiúsclas diferentemente de letras minúsculas.

\begin{Shaded}
\begin{Highlighting}[]
\NormalTok{A}
\end{Highlighting}
\end{Shaded}

\begin{verbatim}
## [1] 9
\end{verbatim}

Também podemos salvar letras e palavras dentro de um objeto, mas para
isso devemos colocar a letra ou a palavra em aspas.

\begin{Shaded}
\begin{Highlighting}[]
\NormalTok{b}\OtherTok{\textless{}{-}} \StringTok{"Hoje"}
\NormalTok{c}\OtherTok{\textless{}{-}} \StringTok{"Eu"}
\NormalTok{d}\OtherTok{\textless{}{-}} \StringTok{"Vou aprender R"}
\end{Highlighting}
\end{Shaded}

\begin{Shaded}
\begin{Highlighting}[]
\NormalTok{b}
\end{Highlighting}
\end{Shaded}

\begin{verbatim}
## [1] "Hoje"
\end{verbatim}

\begin{Shaded}
\begin{Highlighting}[]
\NormalTok{c}
\end{Highlighting}
\end{Shaded}

\begin{verbatim}
## [1] "Eu"
\end{verbatim}

\begin{Shaded}
\begin{Highlighting}[]
\NormalTok{d}
\end{Highlighting}
\end{Shaded}

\begin{verbatim}
## [1] "Vou aprender R"
\end{verbatim}

\hypertarget{funuxe7uxf5es}{%
\subsection{Funções}\label{funuxe7uxf5es}}

O R possui diversas função que podemos utilizar parar realizar
diferentes operações. O uso de uma função é feito escrevendo o nome da
função e entre parêntese os argumentos da função, função (argumentos).
Caso precise passar mais de um argumento para função, os argumentos são
separados por vírgula.

\begin{Shaded}
\begin{Highlighting}[]
\FunctionTok{log}\NormalTok{(}\DecValTok{10}\NormalTok{) }\CommentTok{\# calculando o logarítimo natural de 10.}
\end{Highlighting}
\end{Shaded}

\begin{verbatim}
## [1] 2.302585
\end{verbatim}

Agora podemos calcular o logarítmo de 10 na base 2. Note que temos dois
argumentos separados por vírgula.

\begin{Shaded}
\begin{Highlighting}[]
\FunctionTok{log}\NormalTok{(}\DecValTok{10}\NormalTok{,}\DecValTok{2}\NormalTok{)}
\end{Highlighting}
\end{Shaded}

\begin{verbatim}
## [1] 3.321928
\end{verbatim}

A função prod calcula retona o produto de vários números.

\begin{Shaded}
\begin{Highlighting}[]
\FunctionTok{prod}\NormalTok{(}\DecValTok{2}\NormalTok{,}\DecValTok{3}\NormalTok{,}\DecValTok{4}\NormalTok{,}\DecValTok{5}\NormalTok{,}\DecValTok{6}\NormalTok{)}
\end{Highlighting}
\end{Shaded}

\begin{verbatim}
## [1] 720
\end{verbatim}

Temos também a função sqrt que retorna a raíz quadrada de um número.

\begin{Shaded}
\begin{Highlighting}[]
\FunctionTok{sqrt}\NormalTok{(}\DecValTok{360}\NormalTok{)}
\end{Highlighting}
\end{Shaded}

\begin{verbatim}
## [1] 18.97367
\end{verbatim}

A função round serve para indicar quantas cassas decimais queremos
visualizar. Nela passamos um valor ou uma variável que guarda um valor,
e indicamos quantas casa decimais queremos. Aqui iremos guardar o
resultado da raíz quadrada de 360 dentro de um obejto chamado raiz, e
pedir para o R devover o resultado com apenas duas casas decimais.

\begin{Shaded}
\begin{Highlighting}[]
\NormalTok{raiz}\OtherTok{\textless{}{-}} \FunctionTok{sqrt}\NormalTok{(}\DecValTok{360}\NormalTok{)}
\FunctionTok{round}\NormalTok{(raiz,}\DecValTok{2}\NormalTok{)}
\end{Highlighting}
\end{Shaded}

\begin{verbatim}
## [1] 18.97
\end{verbatim}

Podemos também utilizar uma função que indica a classe da nossa
variável.

\begin{Shaded}
\begin{Highlighting}[]
\NormalTok{a}\OtherTok{\textless{}{-}} \DecValTok{10}
\FunctionTok{class}\NormalTok{(a)}
\end{Highlighting}
\end{Shaded}

\begin{verbatim}
## [1] "numeric"
\end{verbatim}

A variável a é uma variável numérica.

\begin{Shaded}
\begin{Highlighting}[]
\NormalTok{b}\OtherTok{\textless{}{-}} \StringTok{"Eu vou aprender R"}
\FunctionTok{class}\NormalTok{(b)}
\end{Highlighting}
\end{Shaded}

\begin{verbatim}
## [1] "character"
\end{verbatim}

O objeto b é uma variável da classe character, pois é composto por
letras ou simpolos.

Existem também as variáveis do tipo lógicas, são aquelas variáveis que
guardam o resultado de uma comparação lógica, e pode ter o valor TRUE ou
FALSE. Vamos fazer um teste lógico, iremos perguntar se a letra z é
igual à 1 e vamos guardar o resultado detro de um obejeto chamado
logica.

\begin{Shaded}
\begin{Highlighting}[]
\NormalTok{logica}\OtherTok{\textless{}{-}} \StringTok{"z"} \SpecialCharTok{==}\DecValTok{1}
\NormalTok{logica}
\end{Highlighting}
\end{Shaded}

\begin{verbatim}
## [1] FALSE
\end{verbatim}

Obtemos um resultado FALSE, dizendo que a leta z não é igual a 1. A
letra z está entre parentese, pq toda letra que não representa um objeto
precisa estar entre aspas para ser interpretada pelo R.

Podemos agora perguntar se a leta z é diferente de 1.

\begin{Shaded}
\begin{Highlighting}[]
\NormalTok{logica2}\OtherTok{\textless{}{-}} \StringTok{"z"} \SpecialCharTok{!=} \DecValTok{1}
\NormalTok{logica2}
\end{Highlighting}
\end{Shaded}

\begin{verbatim}
## [1] TRUE
\end{verbatim}

Além de testar se a igualdade e a diferença entre variáveis, podemos
também testar se 10 é maior que 0 ou se 2 é menor que 5 por exemplo.

\begin{Shaded}
\begin{Highlighting}[]
\DecValTok{10}\SpecialCharTok{\textgreater{}}\DecValTok{0}
\end{Highlighting}
\end{Shaded}

\begin{verbatim}
## [1] TRUE
\end{verbatim}

\begin{Shaded}
\begin{Highlighting}[]
\DecValTok{2}\SpecialCharTok{\textless{}}\DecValTok{5}
\end{Highlighting}
\end{Shaded}

\begin{verbatim}
## [1] TRUE
\end{verbatim}

Agora pra mostrar que o R interpreta letras maiúsculas de maneira
diferente de letras minusculas, vamos fazer um teste de igualdade.

\begin{Shaded}
\begin{Highlighting}[]
\StringTok{"A"}\SpecialCharTok{==}\StringTok{"a"}
\end{Highlighting}
\end{Shaded}

\begin{verbatim}
## [1] FALSE
\end{verbatim}

Como podemos ver, ele não considera A e a como tendo o mesmo valor.
Vamos checar qual é a classe do objeto que guarda um resultado lógico?

\begin{Shaded}
\begin{Highlighting}[]
\NormalTok{logico3}\OtherTok{\textless{}{-}} \StringTok{"A"}\SpecialCharTok{==}\StringTok{"a"}
\FunctionTok{class}\NormalTok{(logico3)}
\end{Highlighting}
\end{Shaded}

\begin{verbatim}
## [1] "logical"
\end{verbatim}

Podemos ver então que o obejeto logico3 é da classe logical.

\hypertarget{muxf3dulo-3---estrutura-e-manipulauxe7uxe3o-de-dados}{%
\section{Módulo 3 - Estrutura e manipulação de
dados}\label{muxf3dulo-3---estrutura-e-manipulauxe7uxe3o-de-dados}}

\hypertarget{buxf4nus---gruxe1ficos}{%
\section{BÔNUS - GRÁFICOS}\label{buxf4nus---gruxe1ficos}}

\end{document}
